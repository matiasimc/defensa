\documentclass[aspectratio=169,10pt]{beamer}

\usetheme[progressbar=frametitle]{metropolis}
\usepackage{appendixnumberbeamer}
\usepackage{tikz}
\usetikzlibrary{positioning}
\usepackage{booktabs}
\usepackage[scale=2]{ccicons}

\usepackage{pgfplots}
\usepgfplotslibrary{dateplot}

\usepackage{xspace}
\newcommand{\themename}{\textbf{\textsc{metropolis}}\xspace}

\usepackage[spanish,es-tabla]{babel}

\usepackage{xcolor}
\definecolor{dkgreen}{rgb}{0,0.6,0}
\definecolor{gray}{rgb}{0.5,0.5,0.5}
\definecolor{mauve}{rgb}{0.58,0,0.82}
\usepackage{listings}
\lstset{%
	numberstyle=\tiny,
	basicstyle=\fontsize{6.5}{8.3}\ttfamily,
	numbersep=15pt,tabsize=4,
	flexiblecolumns=true,
	keywordstyle=\color{blue},
	commentstyle=\color{dkgreen},
	stringstyle=\color{mauve},
	numberstyle=\tiny\color{gray},
	language=Java,
	breaklines=true,
	breakatwhitespace=true,
  showstringspaces=false,
  aboveskip=0.1em,
  belowskip=0.5em,
	morekeywords={*,num,String,var,library,get,set,StringEq,StringHashEq,bool,Top,Bot,<,String@L,String@H,int@L,int@H,bool@H,bool@L} ,
}


\title{DESCLASIFICACIÓN BASADA EN TIPOS EN DART}
\subtitle{IMPLEMENTACIÓN Y ELABORACIÓN DE HERRAMIENTAS DE INFERENCIA}
% \date{\today}
\date{}
\author{Matías Meneses Cortés}
%\institute{Departamento de Ciencias de la Computación}
\titlegraphic{\hfill\includegraphics[height=2.5cm]{logo.png}}

\begin{document}

\maketitle

\begin{frame}{Contenidos}
   \setbeamertemplate{section in toc}[sections numbered]
   \tableofcontents[hideallsubsections]
\end{frame}

\section{Control de flujo de información}

\begin{frame}[fragile]{Protección de confidencialidad}
  \begin{center}
    \only<1>{\includegraphics[width=0.8\textwidth]{images/interaccion.png}}
		\only<2>{\includegraphics[width=0.8\textwidth]{images/auth.png}}
		\only<3>{\includegraphics[width=0.8\textwidth]{images/gdrive.png}}
		\only<4>{\includegraphics[width=0.8\textwidth]{images/pay.png}}
    \only<5>{\includegraphics[width=0.8\textwidth]{images/interaccion2.png}}
  \end{center}
\end{frame}

\begin{frame}[fragile]{Protección de confidencialidad}
  Distintas técnicas de seguridad en distintas capas de comunicación. \pause
  \vspace{0.5cm}
  \begin{columns}[T,onlytextwidth]
    \column{0.6\textwidth}
    \includegraphics[width=0.8\textwidth]{images/e2e.png} \pause
    \column{0.4\textwidth}
    \includegraphics[width=1.0\textwidth]{images/kernel.png}
  \end{columns}
\end{frame}

\begin{frame}[fragile]{Seguridad basada en el lenguaje: Tipado de seguridad}
	\begin{columns}[T,onlytextwidth]
		\column{0.5\textwidth}
		\includegraphics[width=0.8\textwidth]{images/lbs.png}
		\column{0.5\textwidth}
		\begin{lstlisting}[basicstyle=\fontsize{6.3}{8}\ttfamily]
String@L login(String@L guess, String@H password) {
  if (password == guess) return "Login successful";
  else return "Login failed";
}
    \end{lstlisting}
		\begin{center}
			\begin{tikzpicture}
				\node(H) 												{\texttt{H}};
				\node(L)      [below of=H]       {\texttt{L}};
				\draw(H)      -- (L);
			\end{tikzpicture} \\
			Orden parcial de dos niveles
		\end{center}
	\end{columns}
\end{frame}

\begin{frame}[fragile]{Control de flujo de información}
  \begin{columns}[T,onlytextwidth]
    \column{0.25\textwidth}
    \includegraphics[width=1.0\textwidth]{images/book.png}
    \column{0.75\textwidth}
    \vspace{1cm}
    \begin{onlyenv}<1>
      \begin{lstlisting}
        String book(String username, int date, int cardNumber) {
          return sendToHotel(username, date, cardNumber);
        }

        String sendToHotel(String username, int date, int cardNumber);
        String sendToGoogle(String token, int xCoord, int yCoord);
      \end{lstlisting}
    \end{onlyenv}
    \begin{onlyenv}<2>
      \begin{lstlisting}[escapechar=?]
        String book(String username, int date, int cardNumber) {
          return ?\colorbox{yellow!50}{sendToGoogle}?(username, date, cardNumber);
        }

        String sendToHotel(String username, int date, int cardNumber);
        String sendToGoogle(String token, int xCoord, int yCoord);
      \end{lstlisting}
    \end{onlyenv}

  \end{columns}
\end{frame}

\begin{frame}[fragile]{Tipado de seguridad para el control de flujo de información}

  \begin{columns}[T,onlytextwidth]
    \column{0.25\textwidth}
    \includegraphics[width=1.0\textwidth]{images/book.png}
    \column{0.75\textwidth}
    \vspace{1cm}
    \begin{onlyenv}<1>
      \begin{lstlisting}
        String@L book(String@L username, int@L date, int@H cardNumber) {
          return sendToGoogle(username, date, cardNumber);
        }

        String@L sendToHotel(String@L username, int@L date, int@H cardNumber);
        String@L sendToGoogle(String@H token, int@L xCoord, int@L yCoord);
      \end{lstlisting}
    \end{onlyenv}
    \begin{onlyenv}<2>
      \begin{lstlisting}[escapechar=?]
        String@L book(String@L username, int@L date, int@H cardNumber) {
          ?\colorbox{red!20}{return sendToGoogle(username, date, cardNumber);}?
        }

        String@L sendToHotel(String@L username, int@L date, int@H cardNumber);
        String@L sendToGoogle(String@H token, int@L xCoord, int@L yCoord);
      \end{lstlisting}
    \end{onlyenv}
		\begin{onlyenv}<3>
      \begin{lstlisting}[escapechar=?]
        String@L book(String@L username, int@L date, int@H cardNumber) {
          return sendToHotel(username, date, cardNumber);
        }

        String@L sendToHotel(String@L username, int@L date, int@H cardNumber);
        String@L sendToGoogle(String@H token, int@L xCoord, int@L yCoord);
      \end{lstlisting}
    \end{onlyenv}
  \end{columns}

\end{frame}

\begin{frame}[fragile]{No-interferencia}
  Propiedad fundamental del control de flujo de información.
	\begin{center}
		\includegraphics[width=0.8\textwidth]{images/noninterference.png}
	\end{center}
\end{frame}

\begin{frame}[fragile]{Problema con no-interferencia}
	\begin{center}
\begin{lstlisting}[basicstyle=\fontsize{9}{9}\ttfamily]
            String@L login(String@L guess, String@H password) {
              if (password == guess) return "Login successful";
              else return "Login failed";
            }
\end{lstlisting}
		\vspace{3cm}
		\alert{¡\textbf{No} cumple con no-interferencia!}
	\end{center}
\end{frame}

\begin{frame}[fragile]{Desclasificación}
	\begin{center}
\begin{lstlisting}[basicstyle=\fontsize{9}{9}\ttfamily]
            String@L login(String@L guess, String@H password) {
              if (declassify(password == guess)) return "Login successful";
              else return "Login failed";
            }
\end{lstlisting}
	\end{center}
\end{frame}

\begin{frame}[fragile]{Problema con desclasificación}
	\begin{center}
\begin{lstlisting}[basicstyle=\fontsize{9}{9}\ttfamily]
                                    declassify(password)
\end{lstlisting} \pause
		\vspace{3cm}
		\alert{¡Grave fuga de información!}
	\end{center}
\end{frame}

\begin{frame}[fragile]{Desclasificación basada en tipos}
\begin{lstlisting}[basicstyle=\fontsize{9}{9}\ttfamily]
String<String login(String<String guess, String<StringEq password) {
  if (password.eq(guess)) return "Login successful";
  else return "Login failed";
}
\end{lstlisting} \pause

\begin{itemize}
	\item Tipos de dos facetas \texttt{String<StringEq} \pause
	\item \texttt{StringEq = [eq: String<String -> Bool<Bool]} \pause
	\item \texttt{String <: StringEq} (Tipo bien formado) \pause
	\item No-interferencia relajada
\end{itemize}

\end{frame}

\begin{frame}[fragile]{Retículo de subtipos}
	\begin{center}
		\begin{tikzpicture}[node distance=2.1cm]
			\node(Top) 												{\texttt{Top} $\triangleq [\ ]$};
			\node(StringEq)		[below right=0.7cm and 0.1cm of Top]			{\texttt{StringEq} $\triangleq [\mathtt{eq} : \mathtt{String} \rightarrow \mathtt{Bool}]$};
			\node(StringEqLength)      [below of=StringEq]       {\texttt{StringEqLength} $\triangleq [...,\ \mathtt{length: Unit\rightarrow Int]}$};
			\node(String)				[below of=StringEqLength]       {\texttt{String} $\triangleq [...]$};
			\node(int)					[below left=0.7cm and 0.1cm of Top] 			{\texttt{Int} $\triangleq [\mathtt{abs} : \mathtt{Unit} \rightarrow \mathtt{Int},\ ...]$};
			\draw(Top)      -- (StringEq);
			\draw(Top)      -- (int);
			\draw(StringEq)      -- (StringEqLength);
			\draw(StringEqLength)      -- (String);
		\end{tikzpicture}
	\end{center}
\end{frame}

% \begin{frame}[fragile]{Regla principal de la desclasificación basada en tipos}
% \begin{onlyenv}<1>
% \begin{lstlisting}[escapechar=?,basicstyle=\fontsize{9}{9}\ttfamily]
% String<String login(String<String guess, String<?\colorbox{orange!20}{Top}? password) {
%   if (?\colorbox{orange!20}{password.eq(guess)}?) return "Login successful";
%   else return "Login failed";
% }
% \end{lstlisting}
% \end{onlyenv}
% \begin{onlyenv}<2>
% \begin{lstlisting}[escapechar=?,basicstyle=\fontsize{9}{9}\ttfamily]
% String<String login(String<String guess, String<?\colorbox{orange!20}{Top}? password) {
%   if (?\colorbox{orange!20}{password.eq(guess)}?) ?\colorbox{red!20}{return "Login successful";}?
%   else ?\colorbox{red!20}{return "Login failed";}?
% }
% \end{lstlisting}
% \end{onlyenv}
% \end{frame}

\begin{frame}[fragile]{Problemas con la desclasificación basada en tipos}
	\begin{itemize} \pause
		\item Propuesta sin implementación práctica. \pause
		\item Anotación completa de facetas para realizar análisis. \pause
	\end{itemize}
	\vspace{1cm}
	\begin{onlyenv}<3>
\begin{lstlisting}[basicstyle=\fontsize{9}{9}\ttfamily]
String<String login(String<String guess, String<StringEq password) {
  if (password.eq(guess)) return "Login successful";
  else return "Login failed";
}
\end{lstlisting}

	\end{onlyenv}
	\begin{onlyenv}<4>
\begin{lstlisting}[basicstyle=\fontsize{9}{9}\ttfamily]
String login(String guess, String<StringEq password) {
  if (password.eq(guess)) return "Login successful";
  else return "Login failed";
}
\end{lstlisting}

	\end{onlyenv}
\end{frame}

\begin{frame}[fragile]{Objetivo de la memoria}
	\metroset{block=fill}
	\begin{block}{Objetivo de la memoria}
		Implementar un sistema de inferencia de facetas públicas para la desclasificación basada en tipos, en conjunto con una extensión para ambientes de desarrollo.
	\end{block}
\end{frame}

\section{Inferencia de facetas públicas en Dart}

\begin{frame}[fragile]{Problema de inferencia a resolver}
	Ejemplo 1 \\
	\vspace{1cm}
	\begin{onlyenv}<1>
\begin{lstlisting}[escapechar=?,basicstyle=\fontsize{9}{9}\ttfamily]
bool login(String guess, String password) {
  return password.eq(guess);
}
\end{lstlisting}
	\end{onlyenv}
	\begin{onlyenv}<2>
\begin{lstlisting}[escapechar=?,basicstyle=\fontsize{9}{9}\ttfamily]
bool login(String guess, String<?\alert{StringEq}? password) {
  return password.eq(guess);
}

?\alert{StringEq}? = [eq: String<String -> bool<bool]
\end{lstlisting}
	\end{onlyenv}
\begin{onlyenv}<3>
\begin{lstlisting}[escapechar=?,basicstyle=\fontsize{9}{9}\ttfamily]
bool login(String<?\alert{String}? guess, String<StringEq password) {
  return password.eq(guess);
}
\end{lstlisting}
Faceta pública de métodos que pertenecen a los tipos por defecto: $\mathtt{Bot\rightarrow Bot}$.
	\end{onlyenv}
	\begin{onlyenv}<4>
\begin{lstlisting}[escapechar=?,basicstyle=\fontsize{9}{9}\ttfamily]
bool<?\alert{X}? login(String<String guess, String<StringEq password) {
  return password.eq(guess);
}
\end{lstlisting}
Faceta pública de métodos que pertenecen a los tipos por defecto: $\mathtt{Bot\rightarrow Bot}$.
	\end{onlyenv}
\end{frame}

\begin{frame}[fragile]{Problema de inferencia a resolver}
	Ejemplo 2 \\
	\vspace{1cm}
	\begin{onlyenv}<1>
\begin{lstlisting}[basicstyle=\fontsize{9}{9}\ttfamily]
bool login(String<String guess, String<Top password) {
  return password.eq(guess);
}
\end{lstlisting}
	\end{onlyenv}
	\begin{onlyenv}<2>
\begin{lstlisting}[escapechar=?,basicstyle=\fontsize{9}{9}\ttfamily]
bool<?\alert{Top}? login(String<String guess, String<Top password) {
  return password.eq(guess);
}
\end{lstlisting}
	\end{onlyenv}
\end{frame}

\begin{frame}[fragile]{Lenguaje Dart}
	\begin{center}
		\includegraphics[width=0.75\textwidth]{images/dart.png}
	\end{center}
\end{frame}

\begin{frame}[fragile]{Dart Analyzer}
	\begin{center}
		\includegraphics[width=1.0\textwidth]{images/ast.png}
	\end{center}
\end{frame}

\begin{frame}[fragile]{Analyzer Plugin}
		\begin{center}
      \includegraphics[width=1.0\textwidth]{images/plugin.png}
		\end{center}
\end{frame}

\begin{frame}[fragile]{Subconjunto soportado de Dart}
	\begin{center}
	\begin{lstlisting}[escapechar=!,basicstyle=\fontsize{9}{11}\ttfamily]
class Foo { !\pause!
  String foo(String a, String b) { !\pause!
    String s = "foo"; !\pause!
    if (a == b)!\pause! return a.concat(b);
    return s;
  }
}
  \end{lstlisting}
	\end{center}
\end{frame}

\begin{frame}[fragile]{Declaración de facetas públicas}
	Uso de anotaciones de Dart para declarar las facetas públicas. \\ \pause
	\vspace{1cm}
\begin{lstlisting}[escapechar=?,basicstyle=\fontsize{9}{9}\ttfamily]
bool login(String guess, @S("StringEq") String password) {
  return password.eq(guess);
}
\end{lstlisting}
\end{frame}

\begin{frame}[fragile]{Definición de facetas públicas}
	Uso de clases abstractas de Dart para definir las facetas públicas. \\ \pause
	\vspace{1cm}
\begin{lstlisting}[escapechar=?,basicstyle=\fontsize{9}{9}\ttfamily]
bool login(String guess, @S("StringEq") String password) {
  return password.eq(guess);
}

abstract class StringEq {
  bool eq(String other);
}
\end{lstlisting}
\end{frame}

\begin{frame}[fragile]{Extensión para ambientes de desarrollo}
	\begin{center}
		\includegraphics[width=0.8\textwidth]{images/sequence.pdf}
	\end{center}
\end{frame}

\begin{frame}[fragile]{Tipos de errores}
	Errores de seguridad \\ \pause
	\vspace{1cm}
\begin{lstlisting}[escapechar=?,basicstyle=\fontsize{9}{9}\ttfamily]
bool<bool login(String guess, @S("Top") String password) {
  ?\colorbox{red!20}{return password.eq(guess);}?
}
\end{lstlisting}
\end{frame}

\begin{frame}[fragile]{Tipos de errores}
	Errores de tipo mal formado \\ \pause
	\vspace{1cm}
\begin{lstlisting}[escapechar=?,basicstyle=\fontsize{9}{9}\ttfamily]
bool<Top login(String guess, ?\colorbox{red!20}{@S(\texttt{"}Abs") String password}?) {
  return password.eq(guess);
}

abstract class Abs {
  int abs();
}
\end{lstlisting}
\end{frame}

\begin{frame}[fragile]{Tipos de errores}
	Warning de faceta pública no definida \\ \pause
	\vspace{1cm}
\begin{lstlisting}[escapechar=?,basicstyle=\fontsize{9}{9}\ttfamily]
bool<bool login(String guess, ?\colorbox{yellow!50}{@S("StringEq")}? String password) {
  return password.eq(guess);
}
\end{lstlisting}
\end{frame}

\begin{frame}[fragile]{Tipos de errores}
	Información de faceta pública inferida \\ \pause
	\vspace{1cm}
\begin{lstlisting}[escapechar=?,basicstyle=\fontsize{9}{9}\ttfamily]
bool<bool login(String<String guess, String ?\underline{password}?) {
  return password.eq(guess);
} ?\pause?

password: [eq: String<String -> bool<bool]
\end{lstlisting}
\end{frame}

\begin{frame}[fragile]{Métricas de la implementación}
	\begin{table}
		\caption{Métricas de la implementación}
		\begin{tabular}{c|c}
      Líneas de código & Clases\\
      \hline
      2866 & 42\\
		\end{tabular}
	\end{table}
\end{frame}

\section{Validación}

\begin{frame}[fragile]{Ejemplo: Sistema de autenticación web}
	\begin{center}
		\includegraphics[width=0.5\textwidth]{images/screen4.png}
	\end{center}
\end{frame}

\begin{frame}[fragile]{Ejemplo: Sistema de autenticación web}
	\begin{center}
		\includegraphics[width=0.95\textwidth]{images/database.png}
	\end{center}
\end{frame}

\begin{frame}[fragile]{Ejemplo: Sistema de autenticación web}
	\begin{center}
		\includegraphics[width=1.0\textwidth]{images/login1.png}
	\end{center}
\end{frame}

\begin{frame}[fragile]{Ejemplo: Sistema de autenticación web}
	\begin{center}
		\includegraphics[width=1.0\textwidth]{images/login2.png}
	\end{center}
\end{frame}

\begin{frame}[fragile]{Ejemplo: Sistema de autenticación web}
	\begin{center}
		\includegraphics[width=1.0\textwidth]{images/login25.png}
	\end{center}
\end{frame}

\begin{frame}[fragile]{Ejemplo: Sistema de autenticación web}
	\begin{center}
		\includegraphics[width=1.0\textwidth]{images/login3.png}
	\end{center}
\end{frame}

\begin{frame}[fragile]{Ejemplo: Sistema de autenticación web}
	\begin{center}
		\includegraphics[width=1.0\textwidth]{images/html1.png}
	\end{center}
\end{frame}

\begin{frame}[fragile]{Ejemplo: Sistema de autenticación web}
	\begin{center}
		\includegraphics[width=1.0\textwidth]{images/html2.png}
	\end{center}
\end{frame}

\begin{frame}[fragile]{Ejemplo: Sistema de autenticación web}
	\begin{center}
		\includegraphics[width=1.0\textwidth]{images/html3.png}
	\end{center}
\end{frame}

\begin{frame}[fragile]{Ejemplo: Sistema de autenticación web}
	\begin{center}
		\includegraphics[width=1.0\textwidth]{images/html4.png}
	\end{center}
\end{frame}

\begin{frame}[fragile]{Ejemplo: Sistema de autenticación web}
	\begin{center}
		\includegraphics[width=1.0\textwidth]{images/html0.png}
	\end{center}
\end{frame}

\begin{frame}[fragile]{Ejemplo: Sistema de autenticación web}
	\begin{columns}[T,onlytextwidth]
		\column{0.6\textwidth}
		\begin{center}
			\includegraphics[width=0.5\textwidth]{images/screen4.png}
		\end{center}
		\column{0.6\textwidth}
		\vspace{2cm}
		Código en repositorio GitHub~\cite{repotest}
	\end{columns}

\end{frame}

\begin{frame}[fragile]{Ejemplo: Sistema de autenticación web}
	\begin{table}
		\caption{Anotación de facetas públicas en identificadores}
		\begin{tabular}{c|c}
      Con inferencia & Sin inferencia\\
      \hline
      7 & 22\\
		\end{tabular}
	\end{table}
\end{frame}

\begin{frame}[fragile]{Comprobación de las reglas del sistema de tipos}
	\begin{center}
		\Huge{\alert{14}} \\
		\vspace{1cm}
	\end{center}
	\begin{center}
		Test unitarios en el repositorio del proyecto~\cite{repo}.
	\end{center}

\end{frame}

\section{Conclusiones y trabajo futuro}

\begin{frame}[fragile]{Conclusiones}
	\begin{itemize}
		\item Conexión entre abstracciones de tipo y relaciones de orden de etiquetas de seguridad. \pause
		\item Integración de conceptos de control de flujo de información con infraestructuras existentes. \pause
		\item Implementación para un subconjunto del lenguaje Dart que demuestra la utilidad de desclasificación basada en tipos.
	\end{itemize}

\end{frame}

\begin{frame}[fragile]{Trabajo futuro}
	Formalización de inferencia.
\end{frame}

\begin{frame}[fragile]{Trabajo futuro}
	Extensión al subconjunto soportado de Dart.
\end{frame}

\begin{frame}[fragile]{Trabajo futuro}
	Características de la extensión para ambientes de desarrollo.
\end{frame}

\begin{frame}[fragile]{Bibliografía}

  \bibliography{defensa}
  \bibliographystyle{abbrv}

\end{frame}

{\setbeamercolor{palette primary}{fg=black, bg=yellow}
\begin{frame}[standout]
  Preguntas
\end{frame}
}

\appendix

\begin{frame}[fragile]{Inferencia de tipos en Scala}
	\begin{columns}[T,onlytextwidth]
		\column{0.6\textwidth}
\begin{lstlisting}[language=Scala,basicstyle=\fontsize{9}{9}\ttfamily]
def mathFunction(n1: Int, n2: Float) = {
  val n = fact(n1);
  n + n2;
}

def fact(n: Int) : Int = {
  if (n == 0) return 1
  else n * fact(n-1)
}
\end{lstlisting}
		\column{0.4\textwidth}
		\begin{itemize}
			\item Inferencia de tipos local
			\item Soporte de overloading y conversiones implícitas de tipos
		\end{itemize}
	\end{columns}
\end{frame}

\begin{frame}[fragile]{Inferencia de tipos en OCaml}
	\begin{columns}[T,onlytextwidth]
		\column{0.6\textwidth}
\begin{lstlisting}[language=ML,basicstyle=\fontsize{9}{9}\ttfamily]
# let average a b =
    (a +. b) /. 2.0;;

val average : float -> float -> float
\end{lstlisting}
\column{0.4\textwidth}
\begin{itemize}
	\item Inferencia de tipos global
	\item No soporta overloading y conversiones implícitas
\end{itemize}
	\end{columns}
\end{frame}

\begin{frame}[fragile]{Inferencia de tipos}

	\begin{columns}[T,onlytextwidth]
		\column{0.6\textwidth}
\begin{lstlisting}[basicstyle=\fontsize{9}{9}\ttfamily]
calculate(c, Int n) {
  if (c) return n*2;
  else return n*0.5;
}
\end{lstlisting}
		\column{0.4\textwidth}
		\begin{tikzpicture}[node distance=1.5cm]
	    \node(Top) {\texttt{Top}};
	    \node(num) [below right of=Top]												{\texttt{Num}};
	    \node(int)		[below right of=num]			{\texttt{Int}};
	    \node(float)					[below left of=num] 			{\texttt{Float}};
	    \node(bool) [below left of=Top] {\texttt{Bool}};
	    \draw (Top) -- (num);
	    \draw (Top) -- (bool);
	    \draw (num)      -- (int);
	    \draw (num)      -- (float);
	  \end{tikzpicture}
	\end{columns}
\end{frame}

\begin{frame}[fragile]{Variables de tipo}
	\begin{columns}[T,onlytextwidth]
		\column{0.6\textwidth}
\begin{lstlisting}[basicstyle=\fontsize{9}{9}\ttfamily]
X calculate(Y c, Int n) {
  if (c) return n*2;
  else return n*0.5;
}
\end{lstlisting}
\end{columns}
\end{frame}

\begin{frame}[fragile]{Generación de restricciones}
	\begin{columns}[T,onlytextwidth]
		\column{0.6\textwidth}
\begin{lstlisting}[basicstyle=\fontsize{9}{9}\ttfamily]
X calculate(Y c, Int n) {
  if (c) return n*2;
  else return n*0.5;
}
\end{lstlisting}
		\column{0.4\textwidth}
		\begin{enumerate}
			\item \texttt{Y <: Bool}
			\item \texttt{Int <: X}
			\item \texttt{Float <: X}
		\end{enumerate}
	\end{columns}
\end{frame}

\begin{frame}[fragile]{Encadenamiento de invocaciones a métodos}
	Ejemplo 3 \\
	\vspace{1cm}
	\begin{onlyenv}<1>
\begin{lstlisting}[escapechar=?,basicstyle=\fontsize{9}{9}\ttfamily]
bool<bool login(int<int guess, String password) {
  return password.hash().eq(guess);
}
\end{lstlisting}
	\end{onlyenv}
	\begin{onlyenv}<2>
\begin{lstlisting}[escapechar=?,basicstyle=\fontsize{9}{9}\ttfamily]
bool<bool login(String<String guess, String<?\alert{StringHash}? password) {
  return password.hash().eq(guess);
}

?\alert{StringHash}? = [hash: () -> int< -]
\end{lstlisting}
	\end{onlyenv}
	\begin{onlyenv}<3>
\begin{lstlisting}[escapechar=?,basicstyle=\fontsize{9}{9}\ttfamily]
bool<bool login(String<String guess, String<?\alert{StringHash}? password) {
  return password.hash().eq(guess);
}

?\alert{StringHash}? = [hash: () -> int<?\alert{IntEq}?]
?\alert{IntEq}? = [eq: int<int -> bool<bool]
\end{lstlisting}
	\end{onlyenv}
\end{frame}


\end{document}
